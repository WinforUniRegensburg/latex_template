%%%%%%%%%%%%%%%%%%%%%%%%%%%%%%%%% Einleitung %%%%%%%%%%%%%%%%%%%%%%%%%%%%%%%%%

\chapter{Einleitung}
\label{chap:Einleitung}

Eine Einleitung sollte möglichst präzise die aktuelle Forschungsfrage abbilden und dem
Leser die Zielstellung der Arbeit präsentieren. Durch eine Einordnung in den Gesamtzusammenhang
(inhaltlich/ zeitlich) wird für den Leser eine kompakte Übersicht gestaltet.
Dabei soll vor allem die Motivation, die zur Ausarbeitung der Arbeit geführt hat, dargelegt
werden. Dadurch wird dem Leser sowohl die Relevanz der Forschungsfrage, als auch
ein Grundverständnis der Thematik nahegebracht. Die Verwendung eines Zitates in der
Einleitung kann dabei helfen, die Notwendigkeit bzw. den Zweck der Arbeit verständlich
zu machen.
Im letzten Absatz der Einleitung wird die Struktur der Arbeit beschrieben, sprich: Das jeweilige Vorgehen wird grob beschrieben, sodass der Leser einen Einblick in den Aufbau der Arbeit erhält und sich darüber im Klaren sein kann, was diesen auf den kommenden Seiten erwartet.


%%%%%%%%%%%%%%%%%%%%%%%%%%%%%%%%%%%%%%%%%%%%%%%%%%%%%%%%%%%%%%%%%%%%%%%%%%%%%%
