%%%%%%%%%%%%%%%%%%%%%%%%%%%%%%%%% Forschungsmethode %%%%%%%%%%%%%%%%%%%%%%%%%%%%%%%%%%


\chapter{Forschungsmethode}
\label{chap:Forschungsmethode}


Die Forschungsmethode beschreibt ein planmäßiges und zielgerichtetes Vorgehen\footnote{\url{https://www.uni-erfurt.de/seminarfach/kurs/4/}}. Dessen Anwendung ist essenziell und dadurch wird das menschliche Handeln zum wissenschaftlichen Konstrukt. Ein Autor kann das methodische Vorgehen frei wählen, muss jedoch in seiner Arbeit abwägen, welche Methodik am Besten mit der Zielsetzung seiner Arbeit korreliert.Solche Zugänge können empirsch, analytisch, vergleichend, systematisch, historisch, hermaneutisch, textanalytisch etc. sein \footnote{\url{https://www.uni-muenster.de/imperia/md/content/didaktik_der_chemie/wissenschaftlichesarbeiten/leitfaden.pdf}}. Die dementsprechend gewählte Methodik sollte fundiert im vorhergehenden Teil beschrieben werden und eine Begründung, die zur Wahl dieser Methode geführt hat, sollte in diesem Kapitel aufgeführt sein. 











%%%%%%%%%%%%%%%%%%%%%%%%%%%%%%%%% Forschungsmethode %%%%%%%%%%%%%%%%%%%%%%%%%%%%%%%%%%


