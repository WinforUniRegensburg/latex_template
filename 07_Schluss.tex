%%%%%%%%%%%%%%%%%%%%%%%%%%%%%%%%%% Schluss %%%%%%%%%%%%%%%%%%%%%%%%%%%%%%%%%%%

\chapter{Schluss}
\label{chap:Schluss}


Der Schluss beinhaltet eine Zusammenfassung der Forschungsarbeit. Anders als in der Diskussion, in der auf die Kernergebnisse Bezug genommen wird, wird hier der Einfluss der Arbeit präsentiert. Als einen Indikator für einen präzisen Schluss kann man das Beantworten der Problemstellung, die in der Einleitung formuliert wurde, sehen. Somit wird die Kernfrage der Forschungsarbeit bzw. der Einfluss auf die genannte Problemstellung deutlich und soll dem Leser veranschaulichen inwiefern eine Verbesserung bewirkt worden ist.  


%%%%%%%%%%%%%%%%%%%%%%%%%%%%%%%%%%%%%%%%%%%%%%%%%%%%%%%%%%%%%%%%%%%%%%%%%%%%%%
