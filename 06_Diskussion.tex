%%%%%%%%%%%%%%%%%%%%%%%%%%%%%%%%% Diskussion %%%%%%%%%%%%%%%%%%%%%%%%%%%%%%%%%%

\chapter{Diskussion}
\label{chap:Diskussion}


In der Diskussion werden die gewonnen Ergebnisse behandelt und somit wird ein umfangreiches Fazit gezogen. 
Die Diskussion is dahingehend eine evaluierende Zusammenfassung der Arbeit, die sich auf die Forschungsfrage bezieht. Grundsätzlich kann die Diskussion in drei Blöcke aufgeteilt werden. Zu Beginn sollten die Ergebnisse in Verbindung mit der Forschungsfrage dargelegt werden, die eine Verbesserung bzw. Lösung der aktuellen Situation darstellen. Hierbei können empirische Werte aus dem Hauptteil übernommen werden, die als Grundlage für die Annahme zur Lösung der Frage dienen. Desweiteren sollen einige Grenzen der eigenen Arbeit aufgezeigt werden, die im Rahmen der Forschung aufgetaucht sind und nicht behoben werden konnten. Dies kann beispielsweise eine abgeschächte Evaluation einer Methodik sein, die aufgrund von zeitlichen Restriktionen entstanden ist. Außerdem sollen Vorschläge für die zukünftige Forschung erbracht werden. Dadurch können zukünftige Forschungsarbeiten auf der eigenen Arbeit aufbauen und die erbrachte Denkanstöße bzw. aufgezeigte Probleme analysieren. Zusammengefasst präsentiert die Diskussion die Forschungsergebnisse mit möglichen Restriktionen und Vorschläge für die zukünftige Forschung in der jeweiigen Domäne. 










%%%%%%%%%%%%%%%%%%%%%%%%%%%%%%%%% Diskussion %%%%%%%%%%%%%%%%%%%%%%%%%%%%%%%%%%
