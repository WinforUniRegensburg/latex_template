%
%%%%%%%%%%%%%%%%%%%%%%%%%%%%%%%%%%%%%%%%%%%%%%%%%%%%%%%%%%%
%
% Abstract
%

\thispagestyle{empty}

%Seite zentrieren
\begin{adjustwidth}{-2cm}{}

%Überschrift "Abstract" statt default "Zusammenfassung"
%\renewcommand{\abstractname}{Abstract}

%\begin{abstract}
%\label{abstract}
%Your abstract goes here...

%\blindtext

%\blindtext


\begin{Huge}\textbf{
\newline
\newline
\newline
\newline
\newline %Achtung: hier keine Leerzeile einfügen
Abstract}\end{Huge} \\ \\

Der Abstract wird der eigentlichen Arbeit vorangestellt und dient dazu dem Leser einen kurzen
Überblick über den Forschungshintergrund, den Aufbau, die Methodik, die Datenauswertung und auch
die Ergebnisse zu liefern1
. Meistens umfasst dieser eine Länge von 150 Wörtern. Daher sollte präzise
der eigentliche Zweck der Forschungsarbeit dargestellt werden. Die Relevanz des Abstracts zeigt sich
bei vielen Online Datenbanken, die diesen in einer Vorschau generieren. Dadurch kann der Nutzer
abhängig vom Inhalt entscheiden, ob die Forschungsarbeit für diesen von Bedeutung ist.






%\end{abstract}

\end{adjustwidth}

%%%%%%%%%%%%%%%%%%%%%%%%%%%%%%%%%%%%%%%%%%%%%%%%%%%%%%%%%
%%% Ende Abstract
